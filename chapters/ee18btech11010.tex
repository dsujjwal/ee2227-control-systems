\begin{enumerate}[label=\thesection.\arabic*.,ref=\thesection.\theenumi]
\numberwithin{equation}{enumi} 
\item 
The Transfer function of Phase Lead Compensator is given by \\

\begin{align}
D(s) = \frac{3(s+\frac{1}{3T})}{(s+\frac{1}{T})}
\end{align}

Find out the frequency (in rad/sec), at which $\angle D(j\omega)$ is maximum? \\
\label{prob:ee18btech11010_comp}
\solution
The basic requirement of the phase lead network is that all poles and zeros
of the transfer function of the network must lie on negative real axis
interlacing each other with a zero located as the nearest point to origin.

Substituting $s = j\omega$ in D(s), we get \\

\begin{align}
D(j\omega) = \frac{3(j\omega+\frac{1}{3T})}{(j\omega+\frac{1}{T})}
\end{align}

The phase of this transfer function $\phi(\omega)$ is given by,
\begin{align}
\phi(\omega) = \tan^{-1}(3\omega T)-\tan^{-1}(\omega T)
\end{align}

$\phi(\omega)$ has its maximum at $\omega_c$ Where $\phi '(\omega_c)=0$,

\begin{align}
\phi '(\omega_c) = 0 = \frac{3T}{1+(3\omega _c T)^2}-\frac{T}{1+(\omega _c T)^2}
\end{align}

After solving and Simplification , we have \\

\begin{align}
\omega _c ^2T^2 = \frac{1}{3}
\end{align}

\begin{align}
\omega _c = \sqrt{\frac{1}{3T^2}}
\end{align}

\item Verify your result through a plot.
\\
\solution 
The following plots the Phase value of the transfer function, 

\begin{figure}[htp]
	\centering
	\includegraphics[width=\columnwidth]{./figs/ee18btech11010.eps}
	\caption{}
	\label{fig:ee18btech11010}
\end{figure}
 
\textbf{Applications:}\\ \\
\begin{enumerate}
  \item Phase lead Compensators can be used as High pass filters,Differentiators.
  \item They are used to reduce steady state errors. 
  \item Increases Phase Margin , relative stability.
\end{enumerate}
\item What is purpose of of a Phase Lead Compensator? \\

\solution
 \begin{enumerate}
\item To increase $K_v$ of a system.
\item The slope of the magnitude plot reduces at the gain crossover frequency so that relative stability improves and error decrease due to error is directly proportional to the slope.
\item To increase Phase margin. 
\item To increase the response/speed become as it shifts the gain crossover frequency to a higher value.
\item To decrease the maximum overshoot of the system

\end{enumerate}


\item Through an example, show how the compensator in Problem \ref{prob:ee18btech11010_comp} can be used in a control system. \\
\solution

Consider the following :- \\

 Open loop system $G(s)=\frac{4}{(s)(s+2)}$. Our aim is to cascade this with a phase lead compensator $G_c(s)$ such that we can meet the following performance requirements-  \\

\textbf{Performance requirements for the system:}\\ \\ 
  Steady State:  $K_v$ = \[\lim_{s \to 0} sG(s) = 20 \] \\
  Transient Response: 	$Phase Margin$ $>$ 50$^{\circ}$  \\
			$Gain Margin$ $>$ 10dB \\

\textbf{Analysis of the system with $G_c(s)$ = K yields the following }\\ \\
   For $K_v = 20$ , $K$ = 10 \\
   This leads to: $ Phase margin$ $\approx$ 17$^{\circ}$  \\
                  $ Gain margin$ $\approx$ +$\infty$ dB  \\

\textbf{Design of the lead Compensator:  }\\ \\
Let the Phase lead Compensator be of the form  \\
\begin{align}
G_c(s) = (K_c)(a)\frac{(1+Ts)}{(1+aTS)}
\end{align}
 
\begin{enumerate}
  \item From $K_v$ = 20 ,$K_v = \frac{4aK_c}{2}$  $\to$ $ aK_c = 10$\\
  \item From the Analysis of the system with $G_c(s) = aK_c$ we obtain the additional phase-lead required is: 50\degree $-$ 17\degree $\approx$ 33\degree . We choose lead as 38\degree ($\approx$ 15$\%$ more than 33\degree ) \\
  \item Let $\phi(\omega_m)$ be the maximum phase of $G_c (s)$. After some calculations we arive at the relation \\
    $sin\phi(\omega_m) = \frac{(1-a)}{(1+a)}$ \\
   Substituting $\phi(\omega_m)$ = 38\degree in the above relation , we get the value of $a$ = 0.24 \\
  \item Substituting  $\omega_m$ ,  the frequency with the maximum phase lead angle  and calculating the Magnitude of the Feed Forward Transfer function , we get \\            
  
   $\abs{G{(j\omega_{m})}}$ = $aK_c\abs{\frac{1+jT\omega_m}{{ 1 + jaT\omega_m}}}$ = $aK_c\frac{1}{\sqrt{a}}$ \\
 We choose $\omega_c$ , the new gain crossover frequency so that $\omega_c$ = $\omega_m$ and \\
 $\abs{G{(j\omega_{c})}G_c{(j\omega_{c})}}$ = $1$ \\
 Substituting $a = 24$ , $aK_c = 10$ in the above equation and solving for $\omega_c$ , we get \\
 $\omega_c = 9 rad/sec $

\item This implies For T,
$\omega_c =\frac{1}{T\sqrt{a}}$ = 9 rad/sec which implies $\frac{1}{T}$ = 4.41 ,and \\
 
 $K_c$ = $\frac{20}{2a}$ = 41.7\\ \\

  After substituting the values of $K_c$, T , a we get  $G_c(s) = 41.7 \frac{s+4.41}{s + 18.4}$  
 
\end{enumerate}  

The compensated system is given by: \\ \\


\tikzstyle{block} = [draw, fill=blue!20, rectangle, 
    minimum height=3em, minimum width=6em]
\tikzstyle{sum} = [draw, fill=blue!20, circle, node distance=1cm]
\tikzstyle{input} = [coordinate]
\tikzstyle{output} = [coordinate]
\tikzstyle{pinstyle} = [pin edge={to-,thin,black}]

\begin{tikzpicture}[auto, node distance=3cm,>=latex']
    
    \node [input, name=input] {};
    \node [sum, right of=input] (sum) {};
    \node [block, right of=sum] (controller) {$41.7\frac{s + 4.41}{s+18.4}$};
    \node [block, right of=controller, node distance=3cm] (system) {$\frac{4}{s(s+2)}$};
   
    \draw [->] (controller) -- node[name=u] {$ $} (system);
    \node [output, right of=system] (output) {};
    \node [block, below of=u] (measurements) {1};

    
    \draw [draw,->] (input) -- node {$X(s)$} (sum);
    \draw [->] (sum) -- node {$E(s)$} (controller);
    \draw [->] (system) -- node [name=y] {$Y(s)$}(output);
    \draw [->] (y) |- (measurements);
    \draw [->] (measurements) -| node[pos=0.99] {$-$} 
        node [near end] {$ $} (sum);

\end{tikzpicture}

The effect of the lead compensator is :

\begin{itemize}
  \item  Phase margin : from 17\degree to 50\degree which implies better transient response
with less overshoot.
  \item $\omega_c$ : from 6.3rad/sec to 9 rad/sec which implies the system response is
faster.
\item Gain margin remains $\infty$
\item $K_v = 20$ as required , which is our acceptable steady-state response.
\end{itemize}


                        
  
\end{enumerate}
